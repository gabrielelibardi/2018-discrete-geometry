\documentclass[11pt]{amsart}

\usepackage{a4wide}
\usepackage{paralist}
\usepackage{url}
\usepackage{bbm}
\usepackage{nopageno}

\newcommand{\cA}{\mathcal{A}}
\newcommand{\cS}{\mathcal{S}}
\DeclareMathOperator{\conv}{conv}
\DeclareMathOperator{\New}{New}
\DeclareMathOperator{\area}{area}
\newcommand{\RR}{\mathbbm{R}}
\newcommand{\CC}{\mathbbm{C}}

\begin{document}
\begin{center}
\textbf{\sffamily
   Discrete and Algorithmic Geometry }

\medskip
   Julian Pfeifle,
   UPC, 2018
\end{center}


\begin{center}
  \textbf{\sffamily Sheet 3}

\bigskip
 due on Monday, December 3, 2018

\end{center}

\bigskip
\bigskip
\bigskip

\begin{enumerate}
\item {}[Ziegler, Exercise 6.9] Describe all $4$-polytopes with $7$~vertices. For this, use all the ``visualization tools'' that we have developed so far:
  \begin{itemize}
  \item Schlegel diagrams
  \item Gale diagrams
  \item Combinatorial descriptions (vertex-facet incidence matrices)
  \end{itemize}
  and show how the various types of data correspond to each other.


\end{enumerate}
\end{document}
 