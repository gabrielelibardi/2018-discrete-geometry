\documentclass[10pt,a4paper]{article}
\usepackage[utf8]{inputenc}
\usepackage{amsmath}
\usepackage{amsfonts}
\usepackage{amssymb}
\usepackage{bbm}
\usepackage{dsfont}
\usepackage{amsthm}  %Esto sirve para utilizar simbolos
\usepackage{graphicx}
\usepackage{float}
\usepackage{color}
%\usepackage{subfigure}
\usepackage[left=2cm,right=2cm,top=2cm,bottom=2cm]{geometry}
%\usepackage{caption}
\usepackage{hyperref}
\usepackage{cleveref}
\usepackage{todonotes}
%\usepackage{cancel}
\usepackage[english]{babel}
\usepackage{float}
\usepackage{mathtools}
%\usepackage{subfigure} 
\usepackage{subcaption}
\usepackage{comment}
\usepackage{textcomp}
\usepackage{bm}
\usepackage[nobreak=true]{mdframed}	
%\usepackage{minted}
%\usemintedstyle{mathematica}

\usepackage{fancyhdr}
	\pagestyle{fancy}

\hypersetup{
	colorlinks=false,
	pdfborder={1 1 0.005},
}
	
\newcommand{\cA}{\mathcal{A}}
\newcommand{\cS}{\mathcal{S}}
\DeclareMathOperator{\New}{New}
\DeclareMathOperator{\area}{area}
\DeclareMathOperator{\relint}{relint}
\DeclareMathOperator{\vol}{vol}
\DeclareMathOperator*{\argmin}{arg\,min}
\DeclareMathOperator*{\argmax}{arg\,max}
\def\defs{\stackrel{\tiny{\mbox{def}}}{=}}		% For definitions
	
\DeclareMathOperator{\conv}{conv}
%\renewcommand{\theenumi}{(\alph{enumi})}
\newcommand{\RR}{\mathbb{R}}
\newcommand{\eps}{\varepsilon}
\theoremstyle{plain}
\newmdtheoremenv[linewidth = 1pt]{result}{Result}
\newtheorem*{theorem*}{Theorem}
\newtheorem{theorem}{Theorem}
\newtheorem*{prop}{Proposition}
\newtheorem{lemma}{Lemma}
\newtheorem{corollary}{Corollary}
\theoremstyle{remark}
\newtheorem{fact}{Fact}
\newtheorem{claim}{Claim}
\newtheorem{remark}{Remark}

\theoremstyle{definition}
\newtheorem{definition}{Definition}


\begin{document}
\thispagestyle{plain}
\begin{center}
\rule{\linewidth}{0.05mm}\
{\Large \textbf{Discrete and Algorithmic Geometry: Sheet 4\\}}
{\large Ander Elkoroaristizabal Peleteiro \& Filip Cano Córdoba \& Alberto Larrauri Borroto\\}
\rule{\linewidth}{0.05mm}\
\end{center}

\begin{enumerate}
	\item Definition 9.2 in Ziegler's \emph{Lectures on Polytopes} constructs the linear map
	\[
	P
	\ \xrightarrow{\pi^c}\ 
	Q^c :=
	\Big\{ \binom{\pi(x)}{cx} : x\in P\Big\}
	\ \subset \
	\RR^{q+1}
	\]
	from a projection $\pi:P\subset\RR^p\to Q\subset\RR^q$ and a linear function $c\in(\RR^p)^\star$.
	Is it possible to give an algorithm to determine the set of lower faces 
	$\mathcal L^\downarrow(Q^c)$ of $Q^c$ 
	from just the set of facet normals of~$Q$, 
	the projection~$\pi$, and the linear function~$c$, without running a convex hull algorithm on $Q^c$?
	
	\bigskip\bigskip
	\item Show that
	\[
	\int_P f(x)\,\text{d}x
	\ = \
	\vol(P) \cdot f(p_0)
	\]
	for any polytope $P$ and linear function $f$, 
	where $p_0 = \frac{1}{\vol(P)}\int_P x\,\text{d}x$ denotes the barycenter of $P$.
	\bigskip\bigskip
	\item Complete the proof of Theorem 9.6 in Ziegler's \emph{Lectures on Polytopes}, 
	possibly referring to \cite{bs-1992}.
	
\end{enumerate}


\textbf{1.}

It is not possible to give such an algorithm.

This is because $\pi,c$ and the set of facets of $Q$ do not determine the lower faces of $Q^c$.
Consider $\pi:\RR^2\to \RR$ that deletes the last coordinate. 
Consdier then $c=(0,1)$. 
In this case, $\pi^c$ is the identity in $\RR^2$.
Since in this case $q=1$, the interval is the only polytope the set of facet normals of $q$ is always the same,
so $q=1$, the only relevant information is $\pi$ and $c$, but in this case, $\pi^c$ is the identity.
Therefore, it such algorithm existed, the set of lower faces would be the same for all polygons,
which is not true.

\textbf{2.}

Using the fact that $f$ is linear and linearity of the integral:
\begin{equation}
	\vol(P) f(p_0)
	= \vol(P) f\left( \frac1{\vol (P)}\int_P x\,\mathrm{d}x \right)
	= f\left(\int_P x\,\mathrm{d}x \right) 
	= \int_P f(x) \,\mathrm{d}x
\end{equation}

\textbf{3. }


%\boxed{\textbf{$\bm{\Sigma(P,Q)}$ is a convex set}}
\begin{claim}
	\label{clm:convex}
	\textbf{$\bm{\Sigma(P,Q)}$ is a convex set}
\end{claim}
\begin{proof}
	Consider two points $y_1,y_2 \in \Sigma$,
	and a convex combination of them $y= q_1y_1 + q_2y_2$. 
	Then $y_1 = \int_Q \gamma_1$, $y_2 = \int_Q \gamma_2$ 
	for some $\gamma_1,\gamma_2$ sections. 
	
	Then, by linearity of the integral:
	$y = \int_Q q_1\gamma_1 +q_2\gamma_2$. 
	So we have to see that the convex combination of sections is a section.
	Let $\gamma \defs q_1\gamma_1 +q_2\gamma_2$. 
	Indeed, by linearity of $\pi$:
	\begin{equation}
	\pi(\gamma(x)) = \pi(q_1\gamma_1(x) + q_2\gamma_2(x)) = q_1\pi(\gamma_1(x)) + q_2\pi(\gamma_2(x))
	= (q_1+q_2)x = x
	\end{equation}
\end{proof}
% \boxed{\textbf{$\bm{\Sigma(P,Q) \subseteq \pi^{-1}(r_0)}$}}

\begin{claim}
	\textbf{$\bm{\Sigma(P,Q) \subseteq \pi^{-1}(r_0)}$}
\end{claim}
\begin{proof}
	We want to see that $y\in\Sigma(P,Q) \implies \pi(y) = \bm{r_0}$.
	Consider $y\in \Sigma(P,Q)$. Then there exists $\gamma:Q\to P$ section,
	such that $y =\frac1{\vol(Q)} \int_Q \gamma(x)dx$
	Then:
	\begin{equation}
	\pi(y) = \pi\left(\frac{1}{\vol(Q)} \int_Q \gamma(x)dx\right)
	\end{equation}
	Using linearity of $\pi$ and of the integral, this is equal to
	\footnote{This step probably needs some more explanation. Or maybe not. It's just putting coordinates.}
	\begin{equation}
	\frac{1}{\vol(Q)}\int_Q \pi(\gamma(x))dx 
	= \frac{1}{\vol(Q)}\int_Q xdx 
	= \bm{r_0}
	\end{equation} 	
\end{proof}

\begin{claim}
	$\bm{\mbox{dim}(\Sigma(P,Q)) \leq p-q}$
\end{claim}
\begin{proof}
	To prove this, we only need to prove that 
	$\dim \pi^{-1}(\bm{r_0}) = p-q$. For a section to be tight, 
	its corresponding subdivision must be tight as well, 
	because a non-tight subdivision cannot be the image of a section.
	
	Indeed, since $\pi$ is a linear function, 
	$\dim(\pi^{-1}(\bm{r_0})) = \dim \ker(\pi) = p - \dim \mbox{Im} \pi$.
	Since we are assuming that $P$ and $Q$ have full dimension,
	${\dim\mbox{Im}\pi=q}$.
\end{proof}
\begin{remark}
	A section $\gamma:Q\to P$ is uniquely defined by its image $\gamma (Q)$.
\end{remark}
\begin{proof}
	Given $x\in Q$, $\gamma(x)$ will be the only element in 
	$\pi^{-1}(x)\cap \gamma(Q)$. 
	This set has exactly one element because $\pi\circ \gamma = \mathrm{id}_Q$.
\end{proof}

\begin{definition}
	A section $\gamma:Q\to P$ is \emph{tight}
	\footnote{Because working with tight sections without defining them
		seems to be too \emph{Zieglery}.}
	if:
	\begin{equation}
	\gamma(Q) = \bigcup_{F\in \mathcal F} F \qquad 
	\mbox{for $\mathcal F\subseteq L (P)$ a subset of faces of $P$}		
	\end{equation} 
\end{definition}
\begin{remark}
	For a section to be tight, its corresponding subset of faces $\mathcal{F}$
	must define a $\pi$-induced subdivision of $Q$, that is also tight.
\end{remark}
\begin{proof}
	First observe that a section is an homeomorphism when restricted to its image,
	because it is a continuous function, and its inverse (the restriction of $\pi$)
	is a linear (and thus continuous) map. 
	This means, in particular, that $\gamma$ has to respect dimensions of faces.
	
	For a subset of faces $\mathcal F\in L(P)$ to define a $\pi$-induced subdivision,
	it must satisfy condition (ii) in \cite[Def 9.1]{ziegler2012lectures}.
	Since $\gamma$ maintains dimensions and $\pi$ is a linear projection, 
	for all $F\in \mathcal F$, $\pi^{-1}(\pi(F)) = F$,
	so condition (ii) is always satisfied.
	
	By the same dimensional argument, the $\pi$-induced subdivision of $Q$ 
	defined by $\gamma$ must also be tight. 
\end{proof} 

Note that given $\mathcal F\in L(P)$, the only issue for $\mathcal F$ to 
define a tight section is the part of defining a section, 
because if it does, then it is trivially tight
\footnote{In fact, this observation suggests that the term \textit{tight} is not
	well suited for this kind of sections, but for the sake of clarity, 
	we wanted to use the same naming as in \cite{ziegler2012lectures}.}. 


With this definition, there is trivially a finite number of tight sections,
since each section is defined by its image, 
which is determined by a subset of $L(P)$,
and there are a finite number of them.
%% ALBERTOS PLAYGROUND

\begin{remark} 
	The partial order on $\omega(P,Q)$ 
	defined in \cite[Sec. 9.1]{ziegler2012lectures} 
	is indeed a partial order. 
	In particular, 
	each $\pi$-induced subdivision $\mathcal{F}$ is determined by the union of its faces in $P$, 
	$\bigcup_{F\in \mathcal{F}} F$.
\end{remark}
\begin{proof}
	Let $\mathcal{F}$ be a $\pi$-induced subdivision and let 
	$X= \bigcup_{F\in \mathcal{F}} F$. 
	Let $\mathcal{G}$ be an arbitrary $\pi$-induced subdivision 
	satisfying $ \bigcup_{G\in \mathcal{G}} G=X$. 
	Let $H_1, H_2,...,H_l \subseteq X$ be the maximal elements from $L(P)$ contained in $X$. 
	Then all the $H_i$ must be in $\mathcal{G}$. 
	It is clear that $\pi(H_i)$ must be the maximal faces in $\pi(\mathcal{G})$ 
	and thus $\pi(\mathcal{G})= L(\pi(G_1))\cup \dots \cup L(\pi(G_l))$, 
	as $\pi(\mathcal{G})$ must be a polytopal complex. 
	Finally, condition (ii) in \cite[Def. 9.1]{ziegler2012lectures} 
	implies that the faces in $\mathcal{G}$ must be the ones satisfying 
	$G=\pi^{-1}(J)\cup X$ for some face $J$ of $\pi(\mathcal{G})$, 
	so $\mathcal{G}$ is unequivocally determined by $X$.  
\end{proof}

%Oh ye mighty, upon thy heavy sword our very last hope layeth. Gide our path to break free from the chains of Zieglerity. 
%(Rezar a la deidad de las mates antes de intentar hacer una demostracion larga)

\begin{claim} A $\pi$-section $\gamma$ is not tight if and only if there is a face $F\in L(P)$ such that $F\nsubseteq \gamma(Q)$ and $\relint \, F \cap \gamma(Q)\neq \emptyset$.
\end{claim}
\begin{proof} 
	We will prove the contrapositive statement, i.e.
	$\gamma$ is tight if and only if for every face $F\in L(P)$ such that 
	$\relint \, F \cap \gamma(Q)\neq \emptyset$ 
	then $F\subseteq \gamma(Q)$: \par
	$\boxed{\Rightarrow}$ 
	Suppose that for some $r\in Q$ and $F\in L(P)$, 
	$\gamma(r) \in \relint \,F$.
	Then any face $G\in L(P)$ contains $\gamma(r)$ if and only if $F\leq G$. 
	If $\gamma$ is tight, then $\gamma(Q)$ is an union of faces from $P$, 
	so $\gamma(Q)$ must contain a face greater than $F$ 
	and in consequence it contains $F$ itself. \par
	$\boxed{\Leftarrow}$ 
	Suppose that $\gamma(Q)$ contains all the faces of $P$ whose relative interior intersect. 
	Note that for every $r\in Q$, 
	$\gamma(r)$ belongs to the relative interior of exactly one face of $L(P)$, 
	namely the minimal face containing $\gamma(r)$.
	Let us denote by $F(r)$ to such face. 
	Then, clearly $\gamma(Q)=\bigcup_{r\in Q} F(r)$ and $\gamma$ is tight. 
\end{proof}

We will denote by $L_n(P)$ to the set of faces

\begin{claim} Let $\gamma$ be a $\pi$-section, $r\in Q$ and $F\in L(P)$ such that $\gamma(r) \in  \relint\, F$. If every open set (relative to $Q$) $B\subseteq Q$ satisfying $r\in B$ verifies $\gamma(B)\nsubseteq F$, then there exists a face $G > F$ such that $\gamma(Q)\cap \relint\, G \neq \emptyset$.
\end{claim}
\begin{proof} By hypothesis $\gamma(r)\in \gamma(Q)\setminus$.
\end{proof}

\begin{claim}\label{cl:convexCombSections}
	If a section $\gamma$ is not tight, there exist two sections
	$\gamma_1,\gamma_2$ such that $\gamma$ is a convex combination of 
	$\gamma_1$ and $\gamma_2$, 
	and the three points of $\Sigma(P,Q)$
	defined by them are different. 
\end{claim}
\begin{proof}
	\todo{Coming soon.}
\end{proof}
\begin{claim}
	$\Sigma(P,Q)$ is a polytope.
\end{claim}
\begin{proof}
	We know by \cref{clm:convex} that it is convex.
	By \cref{cl:convexCombSections} and the number of tight sections being finite,
	we know that only a finite number of points cannot be expressed as a convex 
	combination of different elements in $\Sigma(P,Q)$.
	Therefore, it is the convex hull of a finite number of points.
\end{proof}

\begin{remark}
	Now we can say that we can restrict to sections
	that are  piece-wise linear over a subdivision of $Q$,
	because all points of $\Sigma$ are convex combinations
	of points defined by tight sections, 
	that are piece-wise linear over their subdivision of $Q$.
\end{remark}

\begin{definition} 
	Given $S\subset \RR^p$ we will call the \emph{direction}
	$\overrightarrow{S}$ of $S$ to the vector subspace of $\RR^p$ 
	spanned by all vectors of the form $x-y$ for some $x,y\in S$. 
\end{definition}

We will use the following results:

\begin{theorem} 
	\label{thm:maxLinearInPolytope}
	Let $P \subset \RR^p$ be a polytope and $c\in (\RR^ p)^\star$ a linear function. 
	Then $c$ reaches its maximum over $P$ in a non-empty face of $P$. 
	In other words: $\argmax(c_{|_P})\in L(P)$.
\end{theorem}

\begin{remark} In the previous theorem, it is direct that if $F=\argmax(c_{|_P})$ 
	then $\overrightarrow{F}\subseteq \ker c$.
\end{remark}

\begin{definition}
	Given a polytope $P\in \RR^P$, we will say that a linear function $c\in (\RR^p)^\star$ 
	is  \emph{generic} with respect to $P$ if it reaches its maximum exactly in one vertex of $P$, i.e $\argmax(c_{|_P})\in V(P)$.
\end{definition}


\begin{corollary} Let $P\in \RR^P$ be a polytope, and $c\in (\RR^p)^\star$ be a linear function such that for any non-vertex face $F\in L(P)$ it is satisfied $\overrightarrow{F}\nsubseteq \ker c$. Then $c$ is generic respect to $P$.
\end{corollary}

\begin{lemma} 
	Let $P \subset \RR^p$ be a polytope and let $A \subset \RR^p$ be an affine set.
	Then the intersection $P \cap A$ is also a polytope 
	and its non-empty faces are of the form $F\cap A$ for some $F\in L(P)$.
\end{lemma}

\begin{lemma}
	\label{lem:lemilla}
	Given $s\in  {P}$ and $v\in\RR^p$ such that $s+v\in   P$.
	There exists $U\subseteq   P$, 
	an open set (with the topology of $  P$), $s\in U$,
	and $\eps > 0$ such that $U+\eps v \subseteq   P$.
\end{lemma}
\begin{proof}
	\todo{Coming soon.}
\end{proof}

\begin{definition} 
	Given a two polytopes $P\subset \RR^p$, $Q\subset \RR^q$ 
	and a projection between them $\pi: \RR^p \rightarrow \RR^q$, $\pi(P)=Q$, 
	we will say that a linear function $c\in (\RR^p)^\star$ is \textbf{generic} with respect to $\pi$ over $P$ 
	if every face $F\in L(P)$ such that $\overrightarrow{F}\cap \overrightarrow{\ker \pi} \neq \{0\}$
	satisfies $\overrightarrow{F}\cap \overrightarrow{\ker \pi} \nsubseteq \ker c$.
\end{definition}

From now on we will keep the notation used in last definition. 

\begin{corollary} 
	If a linear function $c\in (\RR^ p)^\star$ is generic with respect to 
	$\pi$ over $P$ then it is also generic with respect to each fiber $\pi^ {-1}(r)$, $r\in Q$.
\end{corollary}

%Nos gusta mas la topología que la teoria de la medida. Intentemos evitar usar la segunda:

As its name suggests, "genericness" is an "almost-sure" property:

\begin{lemma} 
	Under the canonical identification $(\RR^ p)^\star \simeq \RR ^p$ 
	the following sets of linear functions in $(\RR^ p)^\star$ are closed with empty interior: 
	\begin{itemize}
		\item[(1)] The set of non-generic functions with respect to $P$.
		\item[(2)] The set of non-generic functions with respect to $\pi$ over $P$.
	\end{itemize}
\end{lemma}
\begin{proof}
	We will prove the statement for case (2). 
	To prove it for case (1) one can proceed analogously.
	Note that given a set $S\subset \RR^p$ and a function 
	$c\in (\RR^p)^\star$, $S\subseteq \ker c$ is equivalent to $c\in S^\perp$. 
	Now, note that there are finitely many linear subspaces  of the form 
	$G=\overrightarrow{F}\cap \overrightarrow{\ker \pi}$ with $G\neq \{0\}$.
	Finally, for any of such $G$'s, 
	$G^\perp$ is trivially closed and it also has empty interior, as $\dim G < p$.  
\end{proof}


%
%\begin{definition}{ALBERTOS DEFINITION}
%	Given an affine map $\pi: \RR^p \rightarrow \RR^q$, a \textbf{generic} linear map $c:\RR ^p %\rightarrow \RR$ with respect to $\pi$ is one satisfying $\ker \, c \cap \, \ker \, %\overline{\pi}= \{0\}$, 
%	where $\overline{\pi}$ is the linear map associated to $\pi$.
%\end{definition}


%We use this concept because vertices of a polytope can be characterized as the points
%of a polytope maximizing a generic function over it.
%\begin{claim}
%	Consider $c\in (\RR^p)^\star$ generic with respect to $\Sigma(P,Q)$ \textcolor{red}{(and %maybe wrt $P$ as well)}. 
%	Then for all $\bm{r}\in Q$, $\pi^{-1}(\bm r)$ has a unique maximal element 
%	with respect to $c$.
%\end{claim}
%\begin{proof}

%	Suppose there existed a fiber 
%	$\pi^{-1}(r)$ 
%	containing two different maximal elements with respect to $c$.
%	Let $a_1, a_2$ be such maximal elements. 
%	Then $ c(a_1) = c(a_2)\iff a_1 - a_2 \in \ker \, c$.
%	And since they are in the same fiber $\pi^{-1}(r)$, 
%	$\pi(a_1) = \pi(a_2) \iff a_1-a_2\in \ker\overline{\pi}$.
%	But then $a_1 - a_2 \in \ker c \cap \ker\overline{\pi} = \{0\}$.
%	In consequence, $a_1=a_2$, a contradiction. \par

%%% Victor Version 
%	TO COMPLETE. Suppose $\pi(x) = Ax + B$ and take $V = \{$subspace of $\mathbb{R}^p$ generated by the rows of A $\} $, then $\pi^{-1}(r) = (p_0 + V^{\perp})\cap P$ with $p_0 \in \pi^{-1}(r)$. Take $c$ generic with respect to $\pi$, then all the vertices of $\pi^{-1}(r)$ have a different image by $c$ and so there is a unique maximal element with respect to $c$. 
%\end{proof}


This way, given $c\in (\RR^p)^\star$ generic with respect to $\pi$ over $P$, we can define a section $\gamma^c$ as
\begin{equation}
\gamma^c(\bm r) = \argmax_{y \in \pi^{-1}(\bm r)}\{ c(y) \}
\end{equation}
\begin{claim} 
	The map $\gamma^c$ is indeed a section.
	%% 	Continuity of gamma^c is somewhat trivial and somewhat not. We can prove it if you want. I think we should. The easiest is to see that it is contunious succession-wise. 
\end{claim}
\begin{proof}
	\todo{Coming soon.}
\end{proof}

\begin{claim}
	The section previously defined $\gamma^c$ is tight,
	and its corresponding subdivision of $Q$ is $\pi$-coherent.
\end{claim}
\begin{proof} 
	%Please. Please. 
	Just note that $S=\{ (x,y) \in \RR^{q+1} \, | \, x\in Q \, , y=c(\gamma^c(x)) \}$ is the union of the lower faces of $Q^c$. 
	This implies that $(\pi^c)^{-1}(S)=\gamma^c(Q)$ is the union of faces of a coherent subdivision.  
\end{proof}


% Comienza la diversion

Now we know that tight coherent subdivisions of $Q$ correspond to vertices of $\Sigma$.
We still have to prove the correspondence of the poset $\omega_{coh}$ with 
the face lattice of $\Sigma(P,Q)$.

To do that, let us first note that every element $c\in(\RR^p)^\star$ defines
both a face in $\Sigma$ and a coherent subdivision in $Q^c$:
\begin{enumerate}
	\item The face it defines is 
	$\phi^c$, given by the valid inequality 
	$c(s) \geq \min_{y\in \Sigma} c(y)$. 
	\item The coherent subdivision it defines is the one given by
	$\mathcal{F}^c$, as in \cite[Def. 9.2]{ziegler2012lectures}.
\end{enumerate}

We will see that we can find a bijection between faces of $\Sigma$ and coherent subdivisions of $Q$
through the elements $c\in(\RR^p)^\star$.

\begin{definition}
	Let $s\in\Sigma$, then
	\begin{equation}
	\Gamma(s)\defs \left\{\gamma:Q\to P \mbox{ section : } 
	s = \frac{1}{\vol(Q)}\int_Q\gamma(x) \,\mathbb{d}x \right\}
	\end{equation}
\end{definition}
\begin{definition}
	Given $c\in(\RR^p)^\star$, let us define the following sets of sections:
	\begin{equation}
	\Gamma(\mathcal{F}^c) \defs 
	\left\{ \gamma\::\: \gamma(Q)\subseteq \bigcup_{F\in \mathcal{F}^c} F \right\}, \qquad
	\Gamma(\phi^c) \defs 
	\bigcup_{s\in\phi^c} \Gamma(s) 
	\end{equation}
\end{definition}
\begin{definition}
	Given $c\in(\RR^p)^\star$, we define the functional 
	$\mathcal{A}^c:\{\mbox{sections of } \pi \} \to \RR$ as:
	\begin{equation}
	\gamma \mapsto \mathcal{A}^c(\gamma) 
	\defs \frac1{\vol(Q)} \int_Q c(\gamma(x))\,\mbox{d}x
	\end{equation}
\end{definition}
\begin{theorem}
	Given $\gamma_1,\gamma_2$ sections, 
	\begin{enumerate}
		\item $\gamma_1,\gamma_2\in\Gamma(\mathcal{F}^c) 
		\implies  \mathcal{A}(\gamma_1) = \mathcal{A}( \gamma_2)$.
		\item $\gamma_1\in\Gamma(\mathcal{F}^c), \gamma_2\notin\Gamma(\mathcal{F}^c)
		\implies \mathcal{A}(\gamma_1) < \mathcal{A}( \gamma_2)$.
	\end{enumerate}
\end{theorem}
\begin{proof}
	$\boxed{1.}$
	
	$\mathcal{A}^c(\gamma_1)-\mathcal{A}^c(\gamma_2) = 
	\frac1{\vol(Q)}\int_Q [c(\gamma_1(x)) - c(\gamma_2(x))]\,\mathrm{d}x$.
	
	Since $\gamma_1,\gamma_2\in\Gamma(\mathcal{F}^c)$,
	we know that $\forall x\in Q$, 
	the points $\binom{x}{c(\gamma_1(x))}$, $\binom{x}{c(\gamma_2(x))}$ 
	are in a lower face of $Q^c$. 
	Since their first $q$ coordinates are equal, and the face is a lower face, 
	it must happen that the last coordinate is also the same. 
	This implies that $[c(\gamma_1(x)) - c(\gamma_2(x))] = 0$.
	Hence $\mathcal{A}^c(\gamma_1)-\mathcal{A}^c(\gamma_2) = 0$.
	
	$\boxed{2. }$
	Since $\gamma_2\notin\Gamma(\mathcal{F}^c)$, 
	there exists $r\in Q$ such that $\gamma(r)$ does not minimize $c$ in $\pi^{-1}(r)$. 
	Let $s \in \argmin\limits_{y\in\pi^{-1}(r)}\{c(y)\}$,
	Then we can define $v\defs s - \gamma(r)$. 
	By construction, we are in the hypothesis of \cref{lem:lemilla}, 
	so there exists $B\subseteq P$ open, and $\eps > 0$, 
	$\gamma_2(r)\in B$ and $B+\eps v \subseteq P$.
	\todo{Completion of proof coming soon}
\end{proof}

% Acaba la diversion

\begin{comment}
	\newpage
	
	Every face $\phi$ in $\Sigma$ is defined by a linear function $c\in(\RR^p)^\star$ and a scalar $c_0$ as:
	\begin{equation}
	\phi = P \cap \{ y \in \RR^p\::\: cy = c_0 \},\qquad
	\mbox{where $cy\geq c_0$ is a valid inequality of $\Sigma$. }
	\end{equation}
	
	\begin{claim}
	Let $c\in(\RR^p)^\star$ and $c_0\in\RR$ be such that they define a face $\phi$ in $\Sigma(P,Q)$.
	Let $\gamma$ be a section, and let $s\in\Sigma(P,Q)$ be the point it defines. Then:
	\begin{equation}
	s\in  \phi \iff \gamma(Q) \subseteq \mathcal F^c \subseteq L(P) 
	\end{equation}
	For $\mathcal F^c$ the preimage via $\pi^c$ of the set of lower faces \hspace{-2mm}
	\footnote{See precise definition of $\mathcal{F}^c$ in \cite[Def. 9.2]{ziegler2012lectures}.}
	of $Q^c$.
	\end{claim}
	\begin{proof}
	$\boxed{ \Leftarrow }$
	
	We want to prove that $c\cdot s=c_0$ and $c\cdot s^\prime \geq c_0$ for all $s^\prime \in \Sigma$.
	Let us number the lower faces of $Q^c$ as 
	$\mathcal L^\downarrow(Q^c) = \{ F_i \}_{i=1:N_c}$.
	
	Each of this $F_i$ is a lower face of $Q^c$. 
	Also, let $Q_i$ be the projection of $F_i$ by the canonical (deleting last coordinate)
	projection from $Q^c$ to $Q$.
	Thus, it is defined by  a linear function $d^i\in(\RR^{q+1})^\star$,
	and a scalar\hspace{-1mm}
	\footnote{In fact, $d_0^i$ is completely defined by $Q^c$ and $d^i$, but we put it for the sake of the argument.}
	$d^i_0\in\RR$.
	
	We can always impose the sign of the scalar, 
	so we impose $d^i_0 \geq 0$. 
	This way, $F_i$ is a lower face if and only if $d_{q+1}^i < 0$.
	
	Given $x\in Q$, by hypothesis $\gamma(x)\in (\pi^c)^{-1}(F_i)$ for some $i$.
	Then 
	\begin{equation}
	\pi^c(\gamma(x)) = \binom{\pi(\gamma(x))}{c\cdot\gamma(x)} = 
	\binom{x}{c\cdot \gamma(x)} \in F_i,
	\end{equation}
	hence the face defining inequality of $F_i$ is verified as an equality:
	\begin{equation}
	d^i\cdot \pi^c(\gamma(x)) 
	= d^i_{[1:q]}\cdot x + d^i_{q+1} \leq d^i_0.
	\end{equation}
	This is equivalent to 
	\begin{equation}\label{eqn:gammageq}
	c\gamma(x) \geq \frac1{d_{q+1}^i}\left(d_0^i - d^i_{[1:q]}\cdot x \right)
	\end{equation}
	Note that we have changed the sign of the inequality because $d_{q+1}^i < 0$. 
	
	%	Since for all $i$, $d_{q+1}^i$ has the same sign, we can define 
	
	With the previous inspiration, we will define our $c_0$ as:
	\begin{equation}
	c_0 \defs\frac1{\vol (Q)} \sum_{i=1}^{N_c} 
	\int_{Q_i}\frac1{d_{q+1}^i}\left(d_0^i - d^i_{[1:q]}\cdot x\right)\, \mathrm{d}x
	\end{equation}
	With this definition of $c_0$, we see that
	\begin{equation}\label{eqn:gamma2}
	c\cdot s = c\left( \frac1{\vol(Q)} \int_Q c\gamma(x)dx \right) =
	\frac1{ \vol(Q)}\sum_{i=1}^{N_c} \int_{Q_i} c\gamma(x)dx
	\end{equation}
	and using \cref{eqn:gammageq}\hspace{-1mm}
	\footnote{Which $s$ verifies as an equality.},
	we get that $c\cdot s = c_0$.
	
	We still have to check that $c\cdot s \geq c_0$ is a valid inequality.
	Indeed, given any section $\gamma^\prime$, 
	\cref{eqn:gammageq} is satisfied in each $Q_i$. 
	Since $Q_i$ form a subdivision of $Q$ and for all $i$,
	the inequalities have the same sign, 
	\cref{eqn:gamma2} together with \cref{eqn:gammageq} yields
	to $c\cdot s \geq c_0$.
	
	$\boxed{ \Rightarrow }$
	
	IN THE NEXT EPISODE
	\end{proof}
	This yields a bijection between the faces of $\Sigma(P,Q)$
	and the coherent subdivisions of $Q$, and thus in particular, 
	between the \emph{coarsest} such subdivisions and the facets of $\Sigma(P,Q)$.
	
	Equivalently, this yields a bijection between the face lattice of $\Sigma(P,Q)$
	and the poset $\omega_{coh}(P,Q)$.
	In particular, this implies that the dimension of $\Sigma(P,Q)$ is indeed $p-q$.
\end{comment}

This last theorem implies that, indeed $\Gamma(\mathcal{F}^c) = \Gamma(\phi^c)$,
which gives the desired bijection between the face lattice of $\Sigma$ and $\omega_{coh}$.


\bibliographystyle{amsplain}
\bibliography{bib}

\end{document}



